\documentclass[notitlepage,12pt]{article}
\usepackage[margin=1in]{geometry}
\usepackage{natbib}
\usepackage{graphicx}
\usepackage{titling}
\usepackage{url}
\graphicspath{{./img}}


\title{%
Fluid Simulation in C++ \\
\large Simulation Assignment}
\author{}
\date{19/04/2021}

\begin{document}
\maketitle

\thispagestyle{empty}

\begin{abstract}
\noindent TODO
\end{abstract}

\newpage
\clearpage
\setcounter{page}{1}

\section{Introduction}
% 40 words

\section{Literature Review}
% 150 words
% \subsection{Jos Stam}

\subsection{Moving Least Squares - Material Point Method}

% very fast and efficient
% very new
% difficult to implement and understand as no existing code to work from

\subsection{Grid-Based Method}

\subsubsection{Vector-Fields}

% briefly what they are and how they are used in fluid sims

\subsubsection{Navier-Stokes \& Jos Stam}

% jos stam/fluid sim for dummies

\section{Implementation}
% 450 words

% watched lots of videos to try to understand the topic as best as I could
% read many articles on implementing different types of fluid simulation
% settled on grid based like jos stam
% used the code on fluid sim for dummies/the coding train then upgraded it to be more object oriented in c++
% and to use the ngl library. full understanding of how the code/algorithm worked was needed to achieve this
% reusing existing code is good programming practice as code should be written to be easily reused - no need to recreate the wheel

% didnt want to do the same dye display as others and instead opted to use Jon's particle system to display.
% this takes velocities at points in the grid and then uses that to drive the particles

\section{Critical Analysis}
% 200 words

\section{Conclusion}
% 40 words

\clearpage
\bibliographystyle{agsm}
\bibliography{report}

\end{document}